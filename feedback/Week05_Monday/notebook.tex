
% Default to the notebook output style

    


% Inherit from the specified cell style.




    
\documentclass[11pt]{article}

    
    
    \usepackage[T1]{fontenc}
    % Nicer default font (+ math font) than Computer Modern for most use cases
    \usepackage{mathpazo}

    % Basic figure setup, for now with no caption control since it's done
    % automatically by Pandoc (which extracts ![](path) syntax from Markdown).
    \usepackage{graphicx}
    % We will generate all images so they have a width \maxwidth. This means
    % that they will get their normal width if they fit onto the page, but
    % are scaled down if they would overflow the margins.
    \makeatletter
    \def\maxwidth{\ifdim\Gin@nat@width>\linewidth\linewidth
    \else\Gin@nat@width\fi}
    \makeatother
    \let\Oldincludegraphics\includegraphics
    % Set max figure width to be 80% of text width, for now hardcoded.
    \renewcommand{\includegraphics}[1]{\Oldincludegraphics[width=.8\maxwidth]{#1}}
    % Ensure that by default, figures have no caption (until we provide a
    % proper Figure object with a Caption API and a way to capture that
    % in the conversion process - todo).
    \usepackage{caption}
    \DeclareCaptionLabelFormat{nolabel}{}
    \captionsetup{labelformat=nolabel}

    \usepackage{adjustbox} % Used to constrain images to a maximum size 
    \usepackage{xcolor} % Allow colors to be defined
    \usepackage{enumerate} % Needed for markdown enumerations to work
    \usepackage{geometry} % Used to adjust the document margins
    \usepackage{amsmath} % Equations
    \usepackage{amssymb} % Equations
    \usepackage{textcomp} % defines textquotesingle
    % Hack from http://tex.stackexchange.com/a/47451/13684:
    \AtBeginDocument{%
        \def\PYZsq{\textquotesingle}% Upright quotes in Pygmentized code
    }
    \usepackage{upquote} % Upright quotes for verbatim code
    \usepackage{eurosym} % defines \euro
    \usepackage[mathletters]{ucs} % Extended unicode (utf-8) support
    \usepackage[utf8x]{inputenc} % Allow utf-8 characters in the tex document
    \usepackage{fancyvrb} % verbatim replacement that allows latex
    \usepackage{grffile} % extends the file name processing of package graphics 
                         % to support a larger range 
    % The hyperref package gives us a pdf with properly built
    % internal navigation ('pdf bookmarks' for the table of contents,
    % internal cross-reference links, web links for URLs, etc.)
    \usepackage{hyperref}
    \usepackage{longtable} % longtable support required by pandoc >1.10
    \usepackage{booktabs}  % table support for pandoc > 1.12.2
    \usepackage[inline]{enumitem} % IRkernel/repr support (it uses the enumerate* environment)
    \usepackage[normalem]{ulem} % ulem is needed to support strikethroughs (\sout)
                                % normalem makes italics be italics, not underlines
    

    
    
    % Colors for the hyperref package
    \definecolor{urlcolor}{rgb}{0,.145,.698}
    \definecolor{linkcolor}{rgb}{.71,0.21,0.01}
    \definecolor{citecolor}{rgb}{.12,.54,.11}

    % ANSI colors
    \definecolor{ansi-black}{HTML}{3E424D}
    \definecolor{ansi-black-intense}{HTML}{282C36}
    \definecolor{ansi-red}{HTML}{E75C58}
    \definecolor{ansi-red-intense}{HTML}{B22B31}
    \definecolor{ansi-green}{HTML}{00A250}
    \definecolor{ansi-green-intense}{HTML}{007427}
    \definecolor{ansi-yellow}{HTML}{DDB62B}
    \definecolor{ansi-yellow-intense}{HTML}{B27D12}
    \definecolor{ansi-blue}{HTML}{208FFB}
    \definecolor{ansi-blue-intense}{HTML}{0065CA}
    \definecolor{ansi-magenta}{HTML}{D160C4}
    \definecolor{ansi-magenta-intense}{HTML}{A03196}
    \definecolor{ansi-cyan}{HTML}{60C6C8}
    \definecolor{ansi-cyan-intense}{HTML}{258F8F}
    \definecolor{ansi-white}{HTML}{C5C1B4}
    \definecolor{ansi-white-intense}{HTML}{A1A6B2}

    % commands and environments needed by pandoc snippets
    % extracted from the output of `pandoc -s`
    \providecommand{\tightlist}{%
      \setlength{\itemsep}{0pt}\setlength{\parskip}{0pt}}
    \DefineVerbatimEnvironment{Highlighting}{Verbatim}{commandchars=\\\{\}}
    % Add ',fontsize=\small' for more characters per line
    \newenvironment{Shaded}{}{}
    \newcommand{\KeywordTok}[1]{\textcolor[rgb]{0.00,0.44,0.13}{\textbf{{#1}}}}
    \newcommand{\DataTypeTok}[1]{\textcolor[rgb]{0.56,0.13,0.00}{{#1}}}
    \newcommand{\DecValTok}[1]{\textcolor[rgb]{0.25,0.63,0.44}{{#1}}}
    \newcommand{\BaseNTok}[1]{\textcolor[rgb]{0.25,0.63,0.44}{{#1}}}
    \newcommand{\FloatTok}[1]{\textcolor[rgb]{0.25,0.63,0.44}{{#1}}}
    \newcommand{\CharTok}[1]{\textcolor[rgb]{0.25,0.44,0.63}{{#1}}}
    \newcommand{\StringTok}[1]{\textcolor[rgb]{0.25,0.44,0.63}{{#1}}}
    \newcommand{\CommentTok}[1]{\textcolor[rgb]{0.38,0.63,0.69}{\textit{{#1}}}}
    \newcommand{\OtherTok}[1]{\textcolor[rgb]{0.00,0.44,0.13}{{#1}}}
    \newcommand{\AlertTok}[1]{\textcolor[rgb]{1.00,0.00,0.00}{\textbf{{#1}}}}
    \newcommand{\FunctionTok}[1]{\textcolor[rgb]{0.02,0.16,0.49}{{#1}}}
    \newcommand{\RegionMarkerTok}[1]{{#1}}
    \newcommand{\ErrorTok}[1]{\textcolor[rgb]{1.00,0.00,0.00}{\textbf{{#1}}}}
    \newcommand{\NormalTok}[1]{{#1}}
    
    % Additional commands for more recent versions of Pandoc
    \newcommand{\ConstantTok}[1]{\textcolor[rgb]{0.53,0.00,0.00}{{#1}}}
    \newcommand{\SpecialCharTok}[1]{\textcolor[rgb]{0.25,0.44,0.63}{{#1}}}
    \newcommand{\VerbatimStringTok}[1]{\textcolor[rgb]{0.25,0.44,0.63}{{#1}}}
    \newcommand{\SpecialStringTok}[1]{\textcolor[rgb]{0.73,0.40,0.53}{{#1}}}
    \newcommand{\ImportTok}[1]{{#1}}
    \newcommand{\DocumentationTok}[1]{\textcolor[rgb]{0.73,0.13,0.13}{\textit{{#1}}}}
    \newcommand{\AnnotationTok}[1]{\textcolor[rgb]{0.38,0.63,0.69}{\textbf{\textit{{#1}}}}}
    \newcommand{\CommentVarTok}[1]{\textcolor[rgb]{0.38,0.63,0.69}{\textbf{\textit{{#1}}}}}
    \newcommand{\VariableTok}[1]{\textcolor[rgb]{0.10,0.09,0.49}{{#1}}}
    \newcommand{\ControlFlowTok}[1]{\textcolor[rgb]{0.00,0.44,0.13}{\textbf{{#1}}}}
    \newcommand{\OperatorTok}[1]{\textcolor[rgb]{0.40,0.40,0.40}{{#1}}}
    \newcommand{\BuiltInTok}[1]{{#1}}
    \newcommand{\ExtensionTok}[1]{{#1}}
    \newcommand{\PreprocessorTok}[1]{\textcolor[rgb]{0.74,0.48,0.00}{{#1}}}
    \newcommand{\AttributeTok}[1]{\textcolor[rgb]{0.49,0.56,0.16}{{#1}}}
    \newcommand{\InformationTok}[1]{\textcolor[rgb]{0.38,0.63,0.69}{\textbf{\textit{{#1}}}}}
    \newcommand{\WarningTok}[1]{\textcolor[rgb]{0.38,0.63,0.69}{\textbf{\textit{{#1}}}}}
    
    
    % Define a nice break command that doesn't care if a line doesn't already
    % exist.
    \def\br{\hspace*{\fill} \\* }
    % Math Jax compatability definitions
    \def\gt{>}
    \def\lt{<}
    % Document parameters
    \title{Week5\_Monday}
    
    
    

    % Pygments definitions
    
\makeatletter
\def\PY@reset{\let\PY@it=\relax \let\PY@bf=\relax%
    \let\PY@ul=\relax \let\PY@tc=\relax%
    \let\PY@bc=\relax \let\PY@ff=\relax}
\def\PY@tok#1{\csname PY@tok@#1\endcsname}
\def\PY@toks#1+{\ifx\relax#1\empty\else%
    \PY@tok{#1}\expandafter\PY@toks\fi}
\def\PY@do#1{\PY@bc{\PY@tc{\PY@ul{%
    \PY@it{\PY@bf{\PY@ff{#1}}}}}}}
\def\PY#1#2{\PY@reset\PY@toks#1+\relax+\PY@do{#2}}

\expandafter\def\csname PY@tok@w\endcsname{\def\PY@tc##1{\textcolor[rgb]{0.73,0.73,0.73}{##1}}}
\expandafter\def\csname PY@tok@c\endcsname{\let\PY@it=\textit\def\PY@tc##1{\textcolor[rgb]{0.25,0.50,0.50}{##1}}}
\expandafter\def\csname PY@tok@cp\endcsname{\def\PY@tc##1{\textcolor[rgb]{0.74,0.48,0.00}{##1}}}
\expandafter\def\csname PY@tok@k\endcsname{\let\PY@bf=\textbf\def\PY@tc##1{\textcolor[rgb]{0.00,0.50,0.00}{##1}}}
\expandafter\def\csname PY@tok@kp\endcsname{\def\PY@tc##1{\textcolor[rgb]{0.00,0.50,0.00}{##1}}}
\expandafter\def\csname PY@tok@kt\endcsname{\def\PY@tc##1{\textcolor[rgb]{0.69,0.00,0.25}{##1}}}
\expandafter\def\csname PY@tok@o\endcsname{\def\PY@tc##1{\textcolor[rgb]{0.40,0.40,0.40}{##1}}}
\expandafter\def\csname PY@tok@ow\endcsname{\let\PY@bf=\textbf\def\PY@tc##1{\textcolor[rgb]{0.67,0.13,1.00}{##1}}}
\expandafter\def\csname PY@tok@nb\endcsname{\def\PY@tc##1{\textcolor[rgb]{0.00,0.50,0.00}{##1}}}
\expandafter\def\csname PY@tok@nf\endcsname{\def\PY@tc##1{\textcolor[rgb]{0.00,0.00,1.00}{##1}}}
\expandafter\def\csname PY@tok@nc\endcsname{\let\PY@bf=\textbf\def\PY@tc##1{\textcolor[rgb]{0.00,0.00,1.00}{##1}}}
\expandafter\def\csname PY@tok@nn\endcsname{\let\PY@bf=\textbf\def\PY@tc##1{\textcolor[rgb]{0.00,0.00,1.00}{##1}}}
\expandafter\def\csname PY@tok@ne\endcsname{\let\PY@bf=\textbf\def\PY@tc##1{\textcolor[rgb]{0.82,0.25,0.23}{##1}}}
\expandafter\def\csname PY@tok@nv\endcsname{\def\PY@tc##1{\textcolor[rgb]{0.10,0.09,0.49}{##1}}}
\expandafter\def\csname PY@tok@no\endcsname{\def\PY@tc##1{\textcolor[rgb]{0.53,0.00,0.00}{##1}}}
\expandafter\def\csname PY@tok@nl\endcsname{\def\PY@tc##1{\textcolor[rgb]{0.63,0.63,0.00}{##1}}}
\expandafter\def\csname PY@tok@ni\endcsname{\let\PY@bf=\textbf\def\PY@tc##1{\textcolor[rgb]{0.60,0.60,0.60}{##1}}}
\expandafter\def\csname PY@tok@na\endcsname{\def\PY@tc##1{\textcolor[rgb]{0.49,0.56,0.16}{##1}}}
\expandafter\def\csname PY@tok@nt\endcsname{\let\PY@bf=\textbf\def\PY@tc##1{\textcolor[rgb]{0.00,0.50,0.00}{##1}}}
\expandafter\def\csname PY@tok@nd\endcsname{\def\PY@tc##1{\textcolor[rgb]{0.67,0.13,1.00}{##1}}}
\expandafter\def\csname PY@tok@s\endcsname{\def\PY@tc##1{\textcolor[rgb]{0.73,0.13,0.13}{##1}}}
\expandafter\def\csname PY@tok@sd\endcsname{\let\PY@it=\textit\def\PY@tc##1{\textcolor[rgb]{0.73,0.13,0.13}{##1}}}
\expandafter\def\csname PY@tok@si\endcsname{\let\PY@bf=\textbf\def\PY@tc##1{\textcolor[rgb]{0.73,0.40,0.53}{##1}}}
\expandafter\def\csname PY@tok@se\endcsname{\let\PY@bf=\textbf\def\PY@tc##1{\textcolor[rgb]{0.73,0.40,0.13}{##1}}}
\expandafter\def\csname PY@tok@sr\endcsname{\def\PY@tc##1{\textcolor[rgb]{0.73,0.40,0.53}{##1}}}
\expandafter\def\csname PY@tok@ss\endcsname{\def\PY@tc##1{\textcolor[rgb]{0.10,0.09,0.49}{##1}}}
\expandafter\def\csname PY@tok@sx\endcsname{\def\PY@tc##1{\textcolor[rgb]{0.00,0.50,0.00}{##1}}}
\expandafter\def\csname PY@tok@m\endcsname{\def\PY@tc##1{\textcolor[rgb]{0.40,0.40,0.40}{##1}}}
\expandafter\def\csname PY@tok@gh\endcsname{\let\PY@bf=\textbf\def\PY@tc##1{\textcolor[rgb]{0.00,0.00,0.50}{##1}}}
\expandafter\def\csname PY@tok@gu\endcsname{\let\PY@bf=\textbf\def\PY@tc##1{\textcolor[rgb]{0.50,0.00,0.50}{##1}}}
\expandafter\def\csname PY@tok@gd\endcsname{\def\PY@tc##1{\textcolor[rgb]{0.63,0.00,0.00}{##1}}}
\expandafter\def\csname PY@tok@gi\endcsname{\def\PY@tc##1{\textcolor[rgb]{0.00,0.63,0.00}{##1}}}
\expandafter\def\csname PY@tok@gr\endcsname{\def\PY@tc##1{\textcolor[rgb]{1.00,0.00,0.00}{##1}}}
\expandafter\def\csname PY@tok@ge\endcsname{\let\PY@it=\textit}
\expandafter\def\csname PY@tok@gs\endcsname{\let\PY@bf=\textbf}
\expandafter\def\csname PY@tok@gp\endcsname{\let\PY@bf=\textbf\def\PY@tc##1{\textcolor[rgb]{0.00,0.00,0.50}{##1}}}
\expandafter\def\csname PY@tok@go\endcsname{\def\PY@tc##1{\textcolor[rgb]{0.53,0.53,0.53}{##1}}}
\expandafter\def\csname PY@tok@gt\endcsname{\def\PY@tc##1{\textcolor[rgb]{0.00,0.27,0.87}{##1}}}
\expandafter\def\csname PY@tok@err\endcsname{\def\PY@bc##1{\setlength{\fboxsep}{0pt}\fcolorbox[rgb]{1.00,0.00,0.00}{1,1,1}{\strut ##1}}}
\expandafter\def\csname PY@tok@kc\endcsname{\let\PY@bf=\textbf\def\PY@tc##1{\textcolor[rgb]{0.00,0.50,0.00}{##1}}}
\expandafter\def\csname PY@tok@kd\endcsname{\let\PY@bf=\textbf\def\PY@tc##1{\textcolor[rgb]{0.00,0.50,0.00}{##1}}}
\expandafter\def\csname PY@tok@kn\endcsname{\let\PY@bf=\textbf\def\PY@tc##1{\textcolor[rgb]{0.00,0.50,0.00}{##1}}}
\expandafter\def\csname PY@tok@kr\endcsname{\let\PY@bf=\textbf\def\PY@tc##1{\textcolor[rgb]{0.00,0.50,0.00}{##1}}}
\expandafter\def\csname PY@tok@bp\endcsname{\def\PY@tc##1{\textcolor[rgb]{0.00,0.50,0.00}{##1}}}
\expandafter\def\csname PY@tok@fm\endcsname{\def\PY@tc##1{\textcolor[rgb]{0.00,0.00,1.00}{##1}}}
\expandafter\def\csname PY@tok@vc\endcsname{\def\PY@tc##1{\textcolor[rgb]{0.10,0.09,0.49}{##1}}}
\expandafter\def\csname PY@tok@vg\endcsname{\def\PY@tc##1{\textcolor[rgb]{0.10,0.09,0.49}{##1}}}
\expandafter\def\csname PY@tok@vi\endcsname{\def\PY@tc##1{\textcolor[rgb]{0.10,0.09,0.49}{##1}}}
\expandafter\def\csname PY@tok@vm\endcsname{\def\PY@tc##1{\textcolor[rgb]{0.10,0.09,0.49}{##1}}}
\expandafter\def\csname PY@tok@sa\endcsname{\def\PY@tc##1{\textcolor[rgb]{0.73,0.13,0.13}{##1}}}
\expandafter\def\csname PY@tok@sb\endcsname{\def\PY@tc##1{\textcolor[rgb]{0.73,0.13,0.13}{##1}}}
\expandafter\def\csname PY@tok@sc\endcsname{\def\PY@tc##1{\textcolor[rgb]{0.73,0.13,0.13}{##1}}}
\expandafter\def\csname PY@tok@dl\endcsname{\def\PY@tc##1{\textcolor[rgb]{0.73,0.13,0.13}{##1}}}
\expandafter\def\csname PY@tok@s2\endcsname{\def\PY@tc##1{\textcolor[rgb]{0.73,0.13,0.13}{##1}}}
\expandafter\def\csname PY@tok@sh\endcsname{\def\PY@tc##1{\textcolor[rgb]{0.73,0.13,0.13}{##1}}}
\expandafter\def\csname PY@tok@s1\endcsname{\def\PY@tc##1{\textcolor[rgb]{0.73,0.13,0.13}{##1}}}
\expandafter\def\csname PY@tok@mb\endcsname{\def\PY@tc##1{\textcolor[rgb]{0.40,0.40,0.40}{##1}}}
\expandafter\def\csname PY@tok@mf\endcsname{\def\PY@tc##1{\textcolor[rgb]{0.40,0.40,0.40}{##1}}}
\expandafter\def\csname PY@tok@mh\endcsname{\def\PY@tc##1{\textcolor[rgb]{0.40,0.40,0.40}{##1}}}
\expandafter\def\csname PY@tok@mi\endcsname{\def\PY@tc##1{\textcolor[rgb]{0.40,0.40,0.40}{##1}}}
\expandafter\def\csname PY@tok@il\endcsname{\def\PY@tc##1{\textcolor[rgb]{0.40,0.40,0.40}{##1}}}
\expandafter\def\csname PY@tok@mo\endcsname{\def\PY@tc##1{\textcolor[rgb]{0.40,0.40,0.40}{##1}}}
\expandafter\def\csname PY@tok@ch\endcsname{\let\PY@it=\textit\def\PY@tc##1{\textcolor[rgb]{0.25,0.50,0.50}{##1}}}
\expandafter\def\csname PY@tok@cm\endcsname{\let\PY@it=\textit\def\PY@tc##1{\textcolor[rgb]{0.25,0.50,0.50}{##1}}}
\expandafter\def\csname PY@tok@cpf\endcsname{\let\PY@it=\textit\def\PY@tc##1{\textcolor[rgb]{0.25,0.50,0.50}{##1}}}
\expandafter\def\csname PY@tok@c1\endcsname{\let\PY@it=\textit\def\PY@tc##1{\textcolor[rgb]{0.25,0.50,0.50}{##1}}}
\expandafter\def\csname PY@tok@cs\endcsname{\let\PY@it=\textit\def\PY@tc##1{\textcolor[rgb]{0.25,0.50,0.50}{##1}}}

\def\PYZbs{\char`\\}
\def\PYZus{\char`\_}
\def\PYZob{\char`\{}
\def\PYZcb{\char`\}}
\def\PYZca{\char`\^}
\def\PYZam{\char`\&}
\def\PYZlt{\char`\<}
\def\PYZgt{\char`\>}
\def\PYZsh{\char`\#}
\def\PYZpc{\char`\%}
\def\PYZdl{\char`\$}
\def\PYZhy{\char`\-}
\def\PYZsq{\char`\'}
\def\PYZdq{\char`\"}
\def\PYZti{\char`\~}
% for compatibility with earlier versions
\def\PYZat{@}
\def\PYZlb{[}
\def\PYZrb{]}
\makeatother


    % Exact colors from NB
    \definecolor{incolor}{rgb}{0.0, 0.0, 0.5}
    \definecolor{outcolor}{rgb}{0.545, 0.0, 0.0}



    
    % Prevent overflowing lines due to hard-to-break entities
    \sloppy 
    % Setup hyperref package
    \hypersetup{
      breaklinks=true,  % so long urls are correctly broken across lines
      colorlinks=true,
      urlcolor=urlcolor,
      linkcolor=linkcolor,
      citecolor=citecolor,
      }
    % Slightly bigger margins than the latex defaults
    
    \geometry{verbose,tmargin=1in,bmargin=1in,lmargin=1in,rmargin=1in}
    
    

    \begin{document}
    
    
    \maketitle
    
    

    
    \section{Table of Contents}\label{table-of-contents}

{1~~}Anatomy of the closed-loop control system

{2~~}Proportional (P) Control

{3~~}Example

{4~~}Exercise

    \section{Anatomy of the closed-loop control
system}\label{anatomy-of-the-closed-loop-control-system}

Last week, you were introduced to the "canonical" form of a system under
feedback control, which is shown in the diagram below:

    

    In the diagram above, as labeled, the system has the following key
pieces:

\begin{itemize}
\tightlist
\item
  The \textbf{plant} (labeled G(s) above) is the system, subsystem, or
  process that is controlled by our feedback controller
\item
  The \textbf{controlled output} (labeled y(s) above) is the output
  variable of the \emph{plant} under control of the feedback controller
\item
  The \textbf{forward path} is the transmission path from the sum
  junction all the way to the controlled output.
\item
  The \textbf{feedforward elements} are all control or compensation
  elements in the forward path. In the example above, this is just C(s).
\item
  The ** control signal ** (labeled u(s) above) is the output signal of
  the feedforward elements (C(s)) that is applied to the plant G(s).
\item
  The \textbf{feedback path} is the transmission path from the
  controlled output back to the sum junction.
\item
  The \textbf{feedback elements} (labeled H(s)) relate the \emph{actual}
  controlled output y(s) and the feedback signal read by the controller
  at the sum junction. Most often, H(s) is a sensor that has some gain
  and filtering characteristics that apply in the act of measuring the
  system output.
\item
  The \textbf{reference input} (labeled r(s)) is an external signal
  signal applied to the system. This is usually our "request" for the
  system to do something.
\item
  The \textbf{primary feedback signal} (labeled y(s)H(s)) is a function
  of the controlled output y(s), transformed by the feedback elements
  H(s). This is the signal that is usually actually read by the
  controller.
\item
  The \textbf{error signal} is the difference between the reference
  input and the primary feedback signal.
\item
  \textbf{Negative feedback} implies that the sum junction
  \emph{subtracts} r-yH. \textbf{Positive feedback} would imply that the
  sum junction adds r+yH. Negative feedback is nearly always used in
  feedback control.
\end{itemize}

We saw in the last reading that the canonical feedback control system
has a closed-loop transfer function:

\begin{equation}
\frac{y(s)}{u(s)} = \frac{CG}{1+CGH}
\end{equation}

But what does this really mean? Fundamentally, it means that the
addition of the controller and feedback to the plant \emph{changes the
dynamics of the plant output}. This is precisely what we want, if our
goal is to improve system performance. In simple terms, \(C(s)\) is a
transfer function that "makes decisions" about how hard to "push" the
plant based on the difference between where the system \emph{is} at any
given instant (\(y(s)H(s)\)), and where the operator \emph{wants the
system to be} (\(r(s)\)). In this course, we will focus on variants of
the PID or "proportional-integral-derivative" controller, which has the
following general transfer function:

\begin{equation}
C(s) = \frac{u(s)}{e(s)}= K_p + K_i\frac{1}{s} + K_ds
\end{equation}

The PID controller responds to the error proportionally (P-control),
and/or the error's derivative proportionally (D-control), and/or the
error's integral proportionally (I-control). PID control almost always
includes a P-control element in the form of \(K_p\), but often, \(K_i\)
and/or \(K_d\) can be set to zero depending on the needs of the engineer
and/or the plant. This week, we will focus on how variants of this
general design work \textbf{for simple first and second order systems}.
We'll use simplified examples and the so-called "direct method" of
determining closed-loop system performance (see chapters 3 and 4 of
Schaum's) to build an understanding of how each type of controller works
before diving into some more involved design tools that will allow us to
apply these controllers effectively to more complex systems.

When synthesizing P(I)(D) controllers (which are discussed in your
Schaum's text in chapters 2 and 6, amongst others), we can have a number
of different design goals. Some of these are listed below:

\begin{itemize}
\tightlist
\item
  Above all, we want to ensure system \emph{stability,} defined as the
  closed loop system having eigenvalues with real parts strictly less
  than 0. We also generally want to know \emph{how stable} a system is..
  in other words, how close is it to being unstable?
\item
  If the system (or its dominant pole) is first order, we may want to
  change its time constant.
\item
  If the system (or its dominant complex conjugate pole pair) is second
  order, we may want to specify its natural frequency and/or damping
  ratio.
\item
  If the system (or its dominant complex conjugate pole pair) is second
  order, we may want to specify its
  \href{http://faculty.mercer.edu/jenkins_he/documents/2ndorderresponseMSD.pdf}{rise
  time, settling time, or percent overshoot}.
\item
  We may want the system to achieve a particular performance in tracking
  a a step or ramp input to \(r(s)\).
\item
  We may want the system to achieve a particular performance in
  rejecting an impulse or step \emph{disturbance input} (not shown
  above, but disturbances are usually modeled as appearing as an input
  to a sum junction in between \(C(s)\) and the plant \(G(s)\)).
\item
  We may want the system to achieve particular magnitude and/or phase
  performance when tracking a sinusoidal input of a known frequency on
  \(r(s)\).
\end{itemize}

Our general procedure for analyzing or designing a control system in
this course will go something like this:

\begin{enumerate}
\def\labelenumi{\arabic{enumi}.}
\tightlist
\item
  Determine the equations and/or system transfer function for each
  component of the system (e.g. \(C(s)\),\(G(s)\),\(H(s)\), etc.)
\item
  Formulate a model of the total, controlled system by appropriately
  connecting the system elements using a block diagram. The canonical
  form above is usually, but not always, a good starting point.
\item
  Design our controller, and determine the system response
  characteristics under the chosen control paradigm.
\item
  Validate our predictions using experimental data, if possible.
\end{enumerate}

Much of what we'll discuss roughly follows the material in Chapter 10 of
your Schaum's text. We will deal with how to handle most of these design
goals as needed, but it is helpful to know where we are going. To begin
our journey, let's take a closer look at so-called "P-control," which is
a special case of PID control with \(K_d\) and \(K_i\) set to 0.

    \section{Proportional (P) Control}\label{proportional-p-control}

Proportional control is the simplest form of the PID feedback control
paradigm. This controller 'pushes' on the plant \emph{proportionally} to
the error between the reference input and the plant output. As the
system gets closer to the desired value of the output, the controller
pushes less and less hard. Think of this controller like a "spring" that
drives the plant to the desired output. For a proportional controller,
the controller transfer function is a constant \(K_p\). Formally, we
write:

\begin{equation}
\frac{u(s)}{e(s)}=C(s) = K_p
\end{equation}

Note that \(e(s)\) and \(u(s)\) will often have different units. This
means that the value of \(K_p\) will assume the units needed to make a
dimensional analysis work out. For example, if \(e(s)\) is a distance
error measured in meters, and \(u(s)\) is a voltage input to a motor
measured in Volts, then the controller's proportional constant \(k_p\)
must have units of \(\frac{V}{m}\). Often, as we work through examples
where we are working with "nameless, faceless transfer functions" for
practice, which may not have units, you'll see that we drop units when
discussing system output \(y\), error \(e\), or controller output \(u\).
This doesn't mean that these units aren't important. It just means that
we're omitting them to keep the discussion as general as possible.

    \section{Example}\label{example}

** Given **

Consider a plant \(P(s) = \frac{1}{s+1}\). Design a proportional
controller to make this system track a step input, and achieve
steady-state in less than 1 second. The system uses a sensor that
measures the plant's output \(y(s)\) perfectly with a gain of \(1\).
After designing your controller, find the maximum controller output
\(u(s)\) needed for a unit step input, and find the system's
steady-state error between reference input and system output. \_\_\_

\textbf{Solution}

To decipher the design goal, recall that a first-order system reaches
steady state in roughly 5 time constants. Therefore, we can say that
we'd like our system to have a time constant of 0.2 seconds or less.
Because the problem also states that the sensing in the system is
'perfect,' we can say that \(H(s)=1\). Applying a proportional
controller to this system and substituting into the closed loop transfer
function \(\frac{y(s)}{r(s)}\) yields:

\begin{equation}
\frac{y(s)}{r(s)} = \frac{K_p\frac{1}{s+1}}{1+K_p\frac{1}{s+1}}
\end{equation}

Clearing the fractions in the numerator and denominator by multiplying
the transfer function by \(\frac{s+1}{s+1}\) yields the final equation
for the system's \emph{closed loop transfer function}:

\begin{equation}
G_{cl}(s) = \frac{y(s)}{r(s)} = \frac{K_p}{s+1+K_p}
\end{equation}

Now, we have to determine what value of \(K_p\) will allow us to reach
our design goal of \(\tau<0.2s\). Because we know that
\(\tau=-\frac{1}{a}\), where \(a\) is the system eigenvalue (the root of
its characteristic equation), what we actually want is for a system with
\(a=-5\). Solving the system's closed-loop characteristic equation means
solving:

\begin{equation}
s+1+K_p=0
\end{equation}

This yields a system eigenvalue \(a=-1-K_p\), so for our example, we
know that \(K_p=4\) will satisfy our design requirement. Let's simulate
the closed-loop system's response to a step input on \(r(s)\).

    \begin{Verbatim}[commandchars=\\\{\}]
{\color{incolor}In [{\color{incolor}1}]:} \PY{n}{s} \PY{p}{=} \PY{n}{tf}\PY{p}{(}\PY{l+s}{\PYZsq{}}\PY{l+s}{s\PYZsq{}}\PY{p}{)}\PY{p}{;}
        \PY{n}{t} \PY{p}{=} \PY{l+m+mi}{0}\PY{p}{:}\PY{p}{.}\PY{l+m+mi}{001}\PY{p}{:}\PY{l+m+mf}{1.5}\PY{p}{;}\PY{c}{\PYZpc{}go a little past where we said the system should be at steady state}
        \PY{n}{G1} \PY{p}{=} \PY{l+m+mi}{1}\PY{o}{/}\PY{p}{(}\PY{n}{s}\PY{o}{+}\PY{l+m+mi}{1}\PY{p}{)}\PY{p}{;}
        \PY{n}{Kp} \PY{p}{=} \PY{l+m+mi}{4}\PY{p}{;}
        \PY{n}{Gcl} \PY{p}{=} \PY{n}{G1}\PY{o}{*}\PY{n}{Kp}\PY{o}{/}\PY{p}{(}\PY{l+m+mi}{1}\PY{o}{+}\PY{n}{G1}\PY{o}{*}\PY{n}{Kp}\PY{p}{)}\PY{p}{;}
        
        \PY{p}{[}\PY{n}{ysim}\PY{p}{,}\PY{n}{tsim}\PY{p}{]}\PY{p}{=}\PY{n}{step}\PY{p}{(}\PY{n}{Gcl}\PY{p}{,}\PY{n}{t}\PY{p}{)}\PY{p}{;}\PY{c}{\PYZpc{}simulate without plotting so we store variables.}
        
        \PY{n+nb}{figure}
        \PY{n+nb}{plot}\PY{p}{(}\PY{n}{tsim}\PY{p}{,}\PY{n}{ysim}\PY{p}{,}\PY{l+s}{\PYZsq{}}\PY{l+s}{k\PYZsq{}}\PY{p}{)}
        \PY{n+nb}{xlabel}\PY{p}{(}\PY{l+s}{\PYZsq{}}\PY{l+s}{Time (s)\PYZsq{}}\PY{p}{)}
        \PY{n+nb}{ylabel}\PY{p}{(}\PY{l+s}{\PYZsq{}}\PY{l+s}{System Output (?)\PYZsq{}}\PY{p}{)}
\end{Verbatim}


    \begin{center}
    \adjustimage{max size={0.9\linewidth}{0.9\paperheight}}{output_6_0.png}
    \end{center}
    { \hspace*{\fill} \\}
    
    As you can see, our controlled system does a pretty good job of getting
to steady-state by 1 second, but there's a problem... one of our design
goals was for the system to track the reference input \(r(s)\). Because
\(r(s)\) was a unit step, we might have expected that our system would
reach a steady-state value of 1, but it doesn't! To show how the
system's error progresses over time, we can simply take the constant
value of \(r(s)\) (which in this case was just 1) and subtract the
system's output under closed loop control from \(r(s)\), since the error
signal \(e(s)=r(s)-y(s)H(s)\). Alternatively, we could use some block
diagram algebra to find the transfer function \(\frac{e(s)}{r(s)}\)
directly. I'll leave the details of this to you to try (highly
recommended for practice!), but if we do the reduction, we find that the
transfer function \(\frac{e(s)}{r(s)}\) for our controlled system is:

\begin{equation}
\frac{e(s)}{r(s)} = \frac{s+1}{s+5}
\end{equation}

Applying the final value theorem can get us an answer about how much
steady-state error we will have in response to a unit step input:

\begin{equation}
\lim_{t\rightarrow \infty} e(t) = \lim_{s\rightarrow 0} s e(s) = \lim_{s\rightarrow 0} s\frac{1}{s}\frac{s+1}{s+5} = 0.2
\end{equation}

So it appears that our controller, while it does meet the requirement to
reach steady state within 1 second, doesn't do a great job of tracking
our reference input! Unless we're ok with \(20\%\) tracking error even
for a simple step input, we'll need something else to do better. Let's
plot the system's error response using MATLAB/Octave:

    \begin{Verbatim}[commandchars=\\\{\}]
{\color{incolor}In [{\color{incolor}2}]:} \PY{c}{\PYZpc{}create our TF from e to y}
        \PY{n}{G\PYZus{}e\PYZus{}y} \PY{p}{=} \PY{p}{(}\PY{n}{s}\PY{o}{+}\PY{l+m+mi}{1}\PY{p}{)}\PY{o}{/}\PY{p}{(}\PY{n}{s}\PY{o}{+}\PY{l+m+mi}{5}\PY{p}{)}\PY{p}{;}
        
        \PY{c}{\PYZpc{}now see how it responds to a step input}
        
        \PY{p}{[}\PY{n}{yey}\PY{p}{,}\PY{n}{tey}\PY{p}{]} \PY{p}{=} \PY{n}{step}\PY{p}{(}\PY{n}{G\PYZus{}e\PYZus{}y}\PY{p}{,}\PY{n}{t}\PY{p}{)}\PY{p}{;}
        
        \PY{n+nb}{figure}
        \PY{n+nb}{plot}\PY{p}{(}\PY{n}{t}\PY{p}{,}\PY{n}{yey}\PY{p}{,}\PY{l+s}{\PYZsq{}}\PY{l+s}{k\PYZsq{}}\PY{p}{)}
        \PY{n+nb}{xlabel}\PY{p}{(}\PY{l+s}{\PYZsq{}}\PY{l+s}{Time (s)\PYZsq{}}\PY{p}{)}
        \PY{n+nb}{ylabel}\PY{p}{(}\PY{l+s}{\PYZsq{}}\PY{l+s}{tracking error\PYZsq{}}\PY{p}{)}
\end{Verbatim}


    \begin{center}
    \adjustimage{max size={0.9\linewidth}{0.9\paperheight}}{output_8_0.png}
    \end{center}
    { \hspace*{\fill} \\}
    
    It appears that the error acts as we'd expect. The error signal could
also have been found by simply subtracting \(y\) from \(r\) given our
original step response. To see how much system input would have been
required to achieve this step response from our controller, we need to
plot the step response of the transfer function \(\frac{u(s)}{r(s)}\),
or in this case, we can simply multiply \(e(s)\) by \(K_p\). This
implies that the maximum controller output needed for the unit step is
\(4\) "units." These units could be voltage if our controller drove an
electric motor, pump, solenoid, or similar electromechanical device, for
instance.

    \section{Exercise}\label{exercise}

Design a P-controller for the following system such that it is
overdamped with a dominant (slowest) time constant is at \(\tau=0.25s\).
Simulate its response to a step input, and comment on its steady-state
tracking error. Assume that the system measures its output perfectly
with no scaling-\/- i.e. \(H(s)=1\). Include any hand-calculations and
diagrams in the markdown cell. These can be typeset or scanned.

\(P(s) = \frac{1}{(s+10)(s+1)}\)


    % Add a bibliography block to the postdoc
    
    
    
    \end{document}
